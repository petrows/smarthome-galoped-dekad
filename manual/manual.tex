\documentclass[11pt,a5paper]{article}
\usepackage{tabularx}
\usepackage{graphicx}
\usepackage{pdfpages}
\usepackage{dashundergaps}
\usepackage{geometry}
% \geometry{
%     a5paper,
%     left=10mm,
%     top=10mm,
%  }

%\setlength{\tabcolsep}{2pt}
\renewcommand{\arraystretch}{1.5}

\newcolumntype{b}{X}
\newcolumntype{s}{>{\hsize=.5\hsize}X}

\title{Galoped-dekad user manual}
\date{\today}

\begin{document}
\maketitle

\bigskip
\small{Version 1.0}

\newpage

\section{Introduction}

\textbf{Galoped-dekad} is a smart home controller based on the popular ESP32
microcontroller. It is designed to be a versatile and user-friendly solution for
home automation enthusiasts. Device features include:

\begin{itemize}
\item Use analog automotive gauges to display CO2 level;
\item Wi-Fi connectivity for remote control and monitoring;
\item User-friendly web interface for easy configuration;
\item MQTT support for integration with other smart home systems;
\item All other features provided by Tasmota firmware;
\end{itemize}

Your device specs:

\begin{itemize}
\item Microcontroller: ESP32;
\item CO2 Sensor: SenseAir S8;
\item Gauge Driver: VID6608;
\item Power Supply: 5V USB-C;
\item Additional Sensors: AHT20 (Temperature and Humidity), BMP280 (Pressure);
\end{itemize}

\newpage

\section{Your unique device}

Your device is build epsecially for you, each version is assembled and configured
individually. Therefore, your device may differ from the standard description in
some aspects. Your device is build as follows:

\bigskip

\begin{tabularx}{\textwidth}{lX}
Build for: & \dotfill \\
Build date: & \dotfill \\
Serial number: & \dotfill \\
MAC Address: & \dotfill \\
Front panel: & \dotfill \\
Backlight: & \dotfill \\
\end{tabularx}

\bigskip

Additional notes:

\newpage

\section{Device control}

\begin{figure}
    \centering
    \includegraphics[width=\textwidth]{assets/galoped-dekad-view.pdf}
    \caption{Device view with controls and sensors}
    \label{fig:DeviceView}
\end{figure}

\subsection{Initial configuration}

New device will start in Access Point mode. Connect to the Wi-Fi network
\textbf{Tasmota-xxxx} (where \textbf{xxxx} are the last 4 digits of the MAC address).

There you will be promted to input connection details for your Wi-Fi network. When done,
device will reboot and connect to your Wi-Fi network. You can find the device IP address
in your router DHCP client list. Open a web browser and navigate to the device IP address
to access the web interface.

\subsection{Settings reset}

To reset the device, hold the \textbf{Backlight} button for 45 seconds until the device restarts.
This will reset all settings to factory defaults. Use "Initial configuration" section to reconnect
to your Wi-Fi network.

\subsection{Device settings}

Device uses default Tasmota web interface. For detailed information on how to use
the web interface, please refer to the official Tasmota documentation at
\texttt{https://tasmota.github.io/docs/}.
\medskip

Device can be connected to various smart home systems using MQTT protocol. For detailed information on how to
configure MQTT settings, please refer to the official Tasmota documentation at
\texttt{https://tasmota.github.io/docs/MQTT/}.
\medskip

Device also supports offline mode: without any configuration it will work as a simple CO2 meter with gauge display.


\end{document}
